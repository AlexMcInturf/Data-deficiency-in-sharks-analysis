\documentclass[]{article}
\usepackage{lmodern}
\usepackage{amssymb,amsmath}
\usepackage{ifxetex,ifluatex}
\usepackage{fixltx2e} % provides \textsubscript
\ifnum 0\ifxetex 1\fi\ifluatex 1\fi=0 % if pdftex
  \usepackage[T1]{fontenc}
  \usepackage[utf8]{inputenc}
\else % if luatex or xelatex
  \ifxetex
    \usepackage{mathspec}
  \else
    \usepackage{fontspec}
  \fi
  \defaultfontfeatures{Ligatures=TeX,Scale=MatchLowercase}
\fi
% use upquote if available, for straight quotes in verbatim environments
\IfFileExists{upquote.sty}{\usepackage{upquote}}{}
% use microtype if available
\IfFileExists{microtype.sty}{%
\usepackage{microtype}
\UseMicrotypeSet[protrusion]{basicmath} % disable protrusion for tt fonts
}{}
\usepackage[margin=1in]{geometry}
\usepackage{hyperref}
\hypersetup{unicode=true,
            pdftitle={DataDeficiency\_analysis},
            pdfauthor={Alex},
            pdfborder={0 0 0},
            breaklinks=true}
\urlstyle{same}  % don't use monospace font for urls
\usepackage{color}
\usepackage{fancyvrb}
\newcommand{\VerbBar}{|}
\newcommand{\VERB}{\Verb[commandchars=\\\{\}]}
\DefineVerbatimEnvironment{Highlighting}{Verbatim}{commandchars=\\\{\}}
% Add ',fontsize=\small' for more characters per line
\usepackage{framed}
\definecolor{shadecolor}{RGB}{248,248,248}
\newenvironment{Shaded}{\begin{snugshade}}{\end{snugshade}}
\newcommand{\KeywordTok}[1]{\textcolor[rgb]{0.13,0.29,0.53}{\textbf{#1}}}
\newcommand{\DataTypeTok}[1]{\textcolor[rgb]{0.13,0.29,0.53}{#1}}
\newcommand{\DecValTok}[1]{\textcolor[rgb]{0.00,0.00,0.81}{#1}}
\newcommand{\BaseNTok}[1]{\textcolor[rgb]{0.00,0.00,0.81}{#1}}
\newcommand{\FloatTok}[1]{\textcolor[rgb]{0.00,0.00,0.81}{#1}}
\newcommand{\ConstantTok}[1]{\textcolor[rgb]{0.00,0.00,0.00}{#1}}
\newcommand{\CharTok}[1]{\textcolor[rgb]{0.31,0.60,0.02}{#1}}
\newcommand{\SpecialCharTok}[1]{\textcolor[rgb]{0.00,0.00,0.00}{#1}}
\newcommand{\StringTok}[1]{\textcolor[rgb]{0.31,0.60,0.02}{#1}}
\newcommand{\VerbatimStringTok}[1]{\textcolor[rgb]{0.31,0.60,0.02}{#1}}
\newcommand{\SpecialStringTok}[1]{\textcolor[rgb]{0.31,0.60,0.02}{#1}}
\newcommand{\ImportTok}[1]{#1}
\newcommand{\CommentTok}[1]{\textcolor[rgb]{0.56,0.35,0.01}{\textit{#1}}}
\newcommand{\DocumentationTok}[1]{\textcolor[rgb]{0.56,0.35,0.01}{\textbf{\textit{#1}}}}
\newcommand{\AnnotationTok}[1]{\textcolor[rgb]{0.56,0.35,0.01}{\textbf{\textit{#1}}}}
\newcommand{\CommentVarTok}[1]{\textcolor[rgb]{0.56,0.35,0.01}{\textbf{\textit{#1}}}}
\newcommand{\OtherTok}[1]{\textcolor[rgb]{0.56,0.35,0.01}{#1}}
\newcommand{\FunctionTok}[1]{\textcolor[rgb]{0.00,0.00,0.00}{#1}}
\newcommand{\VariableTok}[1]{\textcolor[rgb]{0.00,0.00,0.00}{#1}}
\newcommand{\ControlFlowTok}[1]{\textcolor[rgb]{0.13,0.29,0.53}{\textbf{#1}}}
\newcommand{\OperatorTok}[1]{\textcolor[rgb]{0.81,0.36,0.00}{\textbf{#1}}}
\newcommand{\BuiltInTok}[1]{#1}
\newcommand{\ExtensionTok}[1]{#1}
\newcommand{\PreprocessorTok}[1]{\textcolor[rgb]{0.56,0.35,0.01}{\textit{#1}}}
\newcommand{\AttributeTok}[1]{\textcolor[rgb]{0.77,0.63,0.00}{#1}}
\newcommand{\RegionMarkerTok}[1]{#1}
\newcommand{\InformationTok}[1]{\textcolor[rgb]{0.56,0.35,0.01}{\textbf{\textit{#1}}}}
\newcommand{\WarningTok}[1]{\textcolor[rgb]{0.56,0.35,0.01}{\textbf{\textit{#1}}}}
\newcommand{\AlertTok}[1]{\textcolor[rgb]{0.94,0.16,0.16}{#1}}
\newcommand{\ErrorTok}[1]{\textcolor[rgb]{0.64,0.00,0.00}{\textbf{#1}}}
\newcommand{\NormalTok}[1]{#1}
\usepackage{graphicx,grffile}
\makeatletter
\def\maxwidth{\ifdim\Gin@nat@width>\linewidth\linewidth\else\Gin@nat@width\fi}
\def\maxheight{\ifdim\Gin@nat@height>\textheight\textheight\else\Gin@nat@height\fi}
\makeatother
% Scale images if necessary, so that they will not overflow the page
% margins by default, and it is still possible to overwrite the defaults
% using explicit options in \includegraphics[width, height, ...]{}
\setkeys{Gin}{width=\maxwidth,height=\maxheight,keepaspectratio}
\IfFileExists{parskip.sty}{%
\usepackage{parskip}
}{% else
\setlength{\parindent}{0pt}
\setlength{\parskip}{6pt plus 2pt minus 1pt}
}
\setlength{\emergencystretch}{3em}  % prevent overfull lines
\providecommand{\tightlist}{%
  \setlength{\itemsep}{0pt}\setlength{\parskip}{0pt}}
\setcounter{secnumdepth}{0}
% Redefines (sub)paragraphs to behave more like sections
\ifx\paragraph\undefined\else
\let\oldparagraph\paragraph
\renewcommand{\paragraph}[1]{\oldparagraph{#1}\mbox{}}
\fi
\ifx\subparagraph\undefined\else
\let\oldsubparagraph\subparagraph
\renewcommand{\subparagraph}[1]{\oldsubparagraph{#1}\mbox{}}
\fi

%%% Use protect on footnotes to avoid problems with footnotes in titles
\let\rmarkdownfootnote\footnote%
\def\footnote{\protect\rmarkdownfootnote}

%%% Change title format to be more compact
\usepackage{titling}

% Create subtitle command for use in maketitle
\newcommand{\subtitle}[1]{
  \posttitle{
    \begin{center}\large#1\end{center}
    }
}

\setlength{\droptitle}{-2em}
  \title{DataDeficiency\_analysis}
  \pretitle{\vspace{\droptitle}\centering\huge}
  \posttitle{\par}
  \author{Alex}
  \preauthor{\centering\large\emph}
  \postauthor{\par}
  \predate{\centering\large\emph}
  \postdate{\par}
  \date{8/29/2019}


\begin{document}
\maketitle

\subsubsection{This is an R Markdown file that will show the annotated
analysis of our data on factors contributing to data deficiency in all
known shark
species.}\label{this-is-an-r-markdown-file-that-will-show-the-annotated-analysis-of-our-data-on-factors-contributing-to-data-deficiency-in-all-known-shark-species.}

\begin{Shaded}
\begin{Highlighting}[]
\KeywordTok{library}\NormalTok{(tidyverse);}\KeywordTok{library}\NormalTok{(lme4);}\KeywordTok{library}\NormalTok{(sjPlot)}
\end{Highlighting}
\end{Shaded}

\begin{verbatim}
## -- Attaching packages -------------------------------------------------------------------------------------- tidyverse 1.2.1 --
\end{verbatim}

\begin{verbatim}
## √ ggplot2 3.2.0     √ purrr   0.2.4
## √ tibble  1.4.2     √ dplyr   0.7.8
## √ tidyr   0.8.0     √ stringr 1.3.1
## √ readr   1.1.1     √ forcats 0.3.0
\end{verbatim}

\begin{verbatim}
## Warning: package 'ggplot2' was built under R version 3.5.2
\end{verbatim}

\begin{verbatim}
## -- Conflicts ----------------------------------------------------------------------------------------- tidyverse_conflicts() --
## x dplyr::filter() masks stats::filter()
## x dplyr::lag()    masks stats::lag()
\end{verbatim}

\begin{verbatim}
## Loading required package: Matrix
\end{verbatim}

\begin{verbatim}
## 
## Attaching package: 'Matrix'
\end{verbatim}

\begin{verbatim}
## The following object is masked from 'package:tidyr':
## 
##     expand
\end{verbatim}

\begin{verbatim}
## Warning in checkMatrixPackageVersion(): Package version inconsistency detected.
## TMB was built with Matrix version 1.2.15
## Current Matrix version is 1.2.14
## Please re-install 'TMB' from source using install.packages('TMB', type = 'source') or ask CRAN for a binary version of 'TMB' matching CRAN's 'Matrix' package
\end{verbatim}

\begin{verbatim}
## Learn more about sjPlot with 'browseVignettes("sjPlot")'.
\end{verbatim}

\begin{Shaded}
\begin{Highlighting}[]
\CommentTok{# Read in data}
\NormalTok{dd <-}\StringTok{ }\KeywordTok{read.csv}\NormalTok{(}\StringTok{"DD_analysis_datafile_2.csv"}\NormalTok{, }\DataTypeTok{stringsAsFactors =} \OtherTok{FALSE}\NormalTok{)}
\NormalTok{dd =}\StringTok{ }\NormalTok{dd[dd}\OperatorTok{$}\NormalTok{Order }\OperatorTok{!=}\StringTok{ ""}\NormalTok{,]}

\CommentTok{# Set NAs to 0}
\NormalTok{dd[}\KeywordTok{is.na}\NormalTok{(dd)] =}\StringTok{ }\DecValTok{0}
\KeywordTok{table}\NormalTok{(dd}\OperatorTok{$}\NormalTok{Data.Deficient)}
\end{Highlighting}
\end{Shaded}

\begin{verbatim}
## 
##   0   1 
## 313 188
\end{verbatim}

\begin{Shaded}
\begin{Highlighting}[]
\CommentTok{# Rename some columns}
\NormalTok{dd =}\StringTok{ }\NormalTok{dd }\OperatorTok\StringTok{ }\KeywordTok{rename}\NormalTok{(}\StringTok{"Rep_Strategy"}\NormalTok{ =}\StringTok{ "X.Reproductive.Strategy"}\NormalTok{,}
                             \StringTok{"Vulnerability"}\NormalTok{ =}\StringTok{ "Vulnerability.."}\NormalTok{,}
                             \StringTok{"Antarctic"}\NormalTok{ =}\StringTok{ "Antartic"}\NormalTok{, }\StringTok{"D1"}\NormalTok{ =}\StringTok{ "Depth.0.200."}\NormalTok{, }\StringTok{"D2"}\NormalTok{=}\StringTok{"Depth.201.1000."}\NormalTok{, }\StringTok{"D3"}\NormalTok{=}\StringTok{"Depth.1000.."}\NormalTok{)}

\CommentTok{# Cleaning up human uses data (binary = yes or no)}
\NormalTok{dd}\OperatorTok{$}\NormalTok{Human.Uses <-}\StringTok{ }\KeywordTok{ifelse}\NormalTok{ (dd}\OperatorTok{$}\NormalTok{Human.Uses }\OperatorTok{==}\StringTok{ ""}\NormalTok{, }\DecValTok{0}\NormalTok{, }\DecValTok{1}\NormalTok{)}
\KeywordTok{table}\NormalTok{(dd}\OperatorTok{$}\NormalTok{Human.Uses) }
\end{Highlighting}
\end{Shaded}

\begin{verbatim}
## 
##   0   1 
## 179 322
\end{verbatim}

\begin{Shaded}
\begin{Highlighting}[]
\NormalTok{dd}\OperatorTok{$}\NormalTok{Human.Uses <-}\StringTok{ }\KeywordTok{factor}\NormalTok{(dd}\OperatorTok{$}\NormalTok{Human.Uses, }\DataTypeTok{levels =} \KeywordTok{c}\NormalTok{(}\StringTok{"0"}\NormalTok{, }\StringTok{"1"}\NormalTok{))}

\CommentTok{#Cleaning up depth data}
\NormalTok{dd}\OperatorTok{$}\NormalTok{D1[dd}\OperatorTok{$}\NormalTok{D1 }\OperatorTok{==}\StringTok{ ""}\NormalTok{] <-}\StringTok{ "0"}\CommentTok{# D1 corresponds to the 0-200 m depth range}
\NormalTok{dd}\OperatorTok{$}\NormalTok{D1 <-}\StringTok{ }\KeywordTok{factor}\NormalTok{(dd}\OperatorTok{$}\NormalTok{D1, }\DataTypeTok{levels =} \KeywordTok{c}\NormalTok{(}\StringTok{"0"}\NormalTok{, }\StringTok{"1"}\NormalTok{))}
\KeywordTok{class}\NormalTok{(dd}\OperatorTok{$}\NormalTok{D1) }
\end{Highlighting}
\end{Shaded}

\begin{verbatim}
## [1] "factor"
\end{verbatim}

\begin{Shaded}
\begin{Highlighting}[]
\NormalTok{dd}\OperatorTok{$}\NormalTok{D2[dd}\OperatorTok{$}\NormalTok{D2 }\OperatorTok{==}\StringTok{ ""}\NormalTok{] <-}\StringTok{ "0"} \CommentTok{#D2 corresponds to the 201-1000 m depth range}
\NormalTok{dd}\OperatorTok{$}\NormalTok{D2 <-}\StringTok{ }\KeywordTok{factor}\NormalTok{(dd}\OperatorTok{$}\NormalTok{D2, }\DataTypeTok{levels =} \KeywordTok{c}\NormalTok{(}\StringTok{"0"}\NormalTok{, }\StringTok{"1"}\NormalTok{))}
\KeywordTok{class}\NormalTok{(dd}\OperatorTok{$}\NormalTok{D2)}
\end{Highlighting}
\end{Shaded}

\begin{verbatim}
## [1] "factor"
\end{verbatim}

\begin{Shaded}
\begin{Highlighting}[]
\NormalTok{dd}\OperatorTok{$}\NormalTok{D3[dd}\OperatorTok{$}\NormalTok{D3 }\OperatorTok{==}\StringTok{ ""}\NormalTok{] <-}\StringTok{ "0"}
\NormalTok{dd}\OperatorTok{$}\NormalTok{D3 <-}\StringTok{ }\KeywordTok{factor}\NormalTok{(dd}\OperatorTok{$}\NormalTok{D3, }\DataTypeTok{levels =} \KeywordTok{c}\NormalTok{(}\StringTok{"0"}\NormalTok{, }\StringTok{"1"}\NormalTok{))}
\KeywordTok{class}\NormalTok{(dd}\OperatorTok{$}\NormalTok{D3)}
\end{Highlighting}
\end{Shaded}

\begin{verbatim}
## [1] "factor"
\end{verbatim}

\begin{Shaded}
\begin{Highlighting}[]
\CommentTok{# Cleaning up fisheries data}
\NormalTok{dd}\OperatorTok{$}\NormalTok{Fisheries[dd}\OperatorTok{$}\NormalTok{Fisheries }\OperatorTok{==}\StringTok{ "NA"}\NormalTok{] <-}\StringTok{ "Unknown"}
\NormalTok{dd}\OperatorTok{$}\NormalTok{Fisheries[dd}\OperatorTok{$}\NormalTok{Fisheries }\OperatorTok{==}\StringTok{ ""}\NormalTok{] <-}\StringTok{ "Unknown"}
\NormalTok{dd}\OperatorTok{$}\NormalTok{Fisheries <-}\StringTok{ }\KeywordTok{factor}\NormalTok{(dd}\OperatorTok{$}\NormalTok{Fisheries, }\DataTypeTok{levels =} \KeywordTok{c}\NormalTok{(}\StringTok{"Unknown"}\NormalTok{, }\StringTok{"0"}\NormalTok{, }\StringTok{"1"}\NormalTok{))}

\CommentTok{# Cleaning up reproductive strategy data}
\NormalTok{dd}\OperatorTok{$}\NormalTok{Rep_Strategy[dd}\OperatorTok{$}\NormalTok{Rep_Strategy }\OperatorTok{==}\StringTok{ ""}\NormalTok{] =}\StringTok{ "Unknown"}
\NormalTok{dd}\OperatorTok{$}\NormalTok{Rep_Strategy =}\StringTok{ }\KeywordTok{factor}\NormalTok{(dd}\OperatorTok{$}\NormalTok{Rep_Strategy, }\DataTypeTok{levels =} \KeywordTok{c}\NormalTok{(}\StringTok{"Unknown"}\NormalTok{, }\StringTok{"Ovoviviparous"}\NormalTok{, }\StringTok{"Viviparous"}\NormalTok{, }\StringTok{"Oviparous"}\NormalTok{))}

\CommentTok{# Removing columns unused in data}
\NormalTok{dd}\OperatorTok{$}\NormalTok{Benthic <-}\StringTok{ }\KeywordTok{as.numeric}\NormalTok{(dd}\OperatorTok{$}\NormalTok{Benthic) }\OperatorTok\StringTok{ }\KeywordTok{replace_na}\NormalTok{(}\DecValTok{0}\NormalTok{)}
\NormalTok{dd}\OperatorTok{$}\NormalTok{Vulnerability <-}\StringTok{ }\KeywordTok{as.numeric}\NormalTok{(dd}\OperatorTok{$}\NormalTok{Vulnerability) }\OperatorTok\StringTok{ }\KeywordTok{replace_na}\NormalTok{(}\DecValTok{0}\NormalTok{)}

\NormalTok{## Check normality of size variable; needs to be log transformed}
\KeywordTok{hist}\NormalTok{(dd}\OperatorTok{$}\NormalTok{Size,}\DecValTok{100}\NormalTok{)}
\end{Highlighting}
\end{Shaded}

\includegraphics{DataDeficiency_analysis_files/figure-latex/unnamed-chunk-1-1.pdf}

\begin{Shaded}
\begin{Highlighting}[]
\NormalTok{dd}\OperatorTok{$}\NormalTok{log_size<-}\StringTok{ }\KeywordTok{log}\NormalTok{(dd}\OperatorTok{$}\NormalTok{Size) }\CommentTok{#added a column with log transformed values, which we will use in the GLM}

\KeywordTok{summary}\NormalTok{(dd)}
\end{Highlighting}
\end{Shaded}

\begin{verbatim}
##     Order              Family             Genus          
##  Length:501         Length:501         Length:501        
##  Class :character   Class :character   Class :character  
##  Mode  :character   Mode  :character   Mode  :character  
##                                                          
##                                                          
##                                                          
##    species          Common.name        Data.Deficient     Deepwater     
##  Length:501         Length:501         Min.   :0.0000   Min.   :0.0000  
##  Class :character   Class :character   1st Qu.:0.0000   1st Qu.:0.0000  
##  Mode  :character   Mode  :character   Median :0.0000   Median :1.0000  
##                                        Mean   :0.3752   Mean   :0.5329  
##                                        3rd Qu.:1.0000   3rd Qu.:1.0000  
##                                        Max.   :1.0000   Max.   :1.0000  
##     Coastal          Pelagic           Benthic       BrackishFreshwater
##  Min.   :0.0000   Min.   :0.00000   Min.   :0.0000   Min.   :0.00000   
##  1st Qu.:0.0000   1st Qu.:0.00000   1st Qu.:0.0000   1st Qu.:0.00000   
##  Median :1.0000   Median :0.00000   Median :1.0000   Median :0.00000   
##  Mean   :0.5908   Mean   :0.07784   Mean   :0.6347   Mean   :0.08383   
##  3rd Qu.:1.0000   3rd Qu.:0.00000   3rd Qu.:1.0000   3rd Qu.:0.00000   
##  Max.   :1.0000   Max.   :1.00000   Max.   :1.0000   Max.   :1.00000   
##     Tropical        Temperate          Global          Pacific     
##  Min.   :0.0000   Min.   :0.0000   Min.   :0.0000   Min.   :0.000  
##  1st Qu.:0.0000   1st Qu.:0.0000   1st Qu.:0.0000   1st Qu.:0.000  
##  Median :0.0000   Median :0.0000   Median :0.0000   Median :0.000  
##  Mean   :0.4132   Mean   :0.3553   Mean   :0.2076   Mean   :0.477  
##  3rd Qu.:1.0000   3rd Qu.:1.0000   3rd Qu.:0.0000   3rd Qu.:1.000  
##  Max.   :1.0000   Max.   :1.0000   Max.   :1.0000   Max.   :1.000  
##     Atlantic          Indian           Arctic           Antarctic
##  Min.   :0.0000   Min.   :0.0000   Min.   :0.000000   Min.   :0  
##  1st Qu.:0.0000   1st Qu.:0.0000   1st Qu.:0.000000   1st Qu.:0  
##  Median :0.0000   Median :0.0000   Median :0.000000   Median :0  
##  Mean   :0.2335   Mean   :0.2695   Mean   :0.003992   Mean   :0  
##  3rd Qu.:0.0000   3rd Qu.:1.0000   3rd Qu.:0.000000   3rd Qu.:0  
##  Max.   :1.0000   Max.   :1.0000   Max.   :1.000000   Max.   :0  
##  Trans.oceanic       Depth           D1      D2      D3      Human.Uses
##  Min.   :0.0000   Length:501         0:224   0:201   0:422   0:179     
##  1st Qu.:0.0000   Class :character   1:277   1:300   1: 79   1:322     
##  Median :0.0000   Mode  :character                                     
##  Mean   :0.1178                                                        
##  3rd Qu.:0.0000                                                        
##  Max.   :1.0000                                                        
##    Fisheries   minor.commercial   commercial      subsistence     
##  Unknown:  0   Min.   :0.0000   Min.   :0.0000   Min.   :0.00000  
##  0      :295   1st Qu.:0.0000   1st Qu.:0.0000   1st Qu.:0.00000  
##  1      :206   Median :0.0000   Median :0.0000   Median :0.00000  
##                Mean   :0.1996   Mean   :0.1317   Mean   :0.06188  
##                3rd Qu.:0.0000   3rd Qu.:0.0000   3rd Qu.:0.00000  
##                Max.   :1.0000   Max.   :1.0000   Max.   :1.00000  
##     Gamefish         Aquarium       Vulnerability        Size       
##  Min.   :0.0000   Min.   :0.00000   Min.   :0.000   Min.   :  16.0  
##  1st Qu.:0.0000   1st Qu.:0.00000   1st Qu.:2.000   1st Qu.:  51.0  
##  Median :0.0000   Median :0.00000   Median :2.500   Median :  80.0  
##  Mean   :0.1377   Mean   :0.03593   Mean   :2.638   Mean   : 119.2  
##  3rd Qu.:0.0000   3rd Qu.:0.00000   3rd Qu.:3.500   3rd Qu.: 130.0  
##  Max.   :1.0000   Max.   :1.00000   Max.   :4.000   Max.   :1900.0  
##         Rep_Strategy    log_size    
##  Unknown      :126   Min.   :2.773  
##  Ovoviviparous:171   1st Qu.:3.932  
##  Viviparous   : 83   Median :4.382  
##  Oviparous    :121   Mean   :4.464  
##                      3rd Qu.:4.868  
##                      Max.   :7.550
\end{verbatim}

\begin{Shaded}
\begin{Highlighting}[]
\CommentTok{#simple look: }
\CommentTok{#Out of 501 shark species assessed: 37.5% are classified as DD, with ~ 4% not evaluated}
\KeywordTok{sum}\NormalTok{(dd}\OperatorTok{$}\NormalTok{Data.Deficient}\OperatorTok{==}\DecValTok{1}\NormalTok{, }\DataTypeTok{na.rm=}\NormalTok{T)}\OperatorTok{/}\KeywordTok{nrow}\NormalTok{(dd) }\CommentTok{#.375}
\end{Highlighting}
\end{Shaded}

\begin{verbatim}
## [1] 0.3752495
\end{verbatim}

\begin{Shaded}
\begin{Highlighting}[]
\KeywordTok{sum}\NormalTok{(dd}\OperatorTok{$}\NormalTok{Data.Deficient}\OperatorTok{==}\DecValTok{0}\NormalTok{, }\DataTypeTok{na.rm=}\NormalTok{T)}\OperatorTok{/}\KeywordTok{nrow}\NormalTok{(dd) }\CommentTok{# .585}
\end{Highlighting}
\end{Shaded}

\begin{verbatim}
## [1] 0.6247505
\end{verbatim}

\subsubsection{After we have our superficial values describing the data,
time to do some stats and visualization. What potential predictive
factors are we interested
in?}\label{after-we-have-our-superficial-values-describing-the-data-time-to-do-some-stats-and-visualization.-what-potential-predictive-factors-are-we-interested-in}

\emph{Biology}\\
Size\\
Reproductive strategy

\emph{Human Use}\\
Human use in general (any use)\\
Type of use (fisheries, gamefish, aquaculture) Fisheries (any fishery)

\emph{Ecology}\\
Geographic range\\
Latitude/temperature (tropical/temperate)\\
Habitat (benthic, deepwater, pelagic, coastal, freshwater)\\
Depth range

\begin{Shaded}
\begin{Highlighting}[]
\CommentTok{#First, we will look at the coefficients of every single first order variable and extract p-values to see what we are dealing with}
\NormalTok{firstorder.mod <-}\StringTok{ }\KeywordTok{glm}\NormalTok{(Data.Deficient }\OperatorTok{~}\StringTok{ }\NormalTok{Deepwater }\OperatorTok{+}\StringTok{ }\NormalTok{Coastal }\OperatorTok{+}\StringTok{ }\NormalTok{Pelagic }\OperatorTok{+}\StringTok{ }\NormalTok{Benthic }\OperatorTok{+}\StringTok{ }\NormalTok{BrackishFreshwater }\OperatorTok{+}
\StringTok{      }\NormalTok{Tropical }\OperatorTok{+}\StringTok{ }\NormalTok{Temperate }\OperatorTok{+}\StringTok{ }
\StringTok{      }\NormalTok{Global }\OperatorTok{+}\StringTok{ }\NormalTok{Pacific }\OperatorTok{+}\StringTok{ }\NormalTok{Atlantic }\OperatorTok{+}\StringTok{ }\NormalTok{Indian }\OperatorTok{+}\StringTok{ }\NormalTok{Arctic }\OperatorTok{+}\StringTok{ }\NormalTok{Antarctic }\OperatorTok{+}\StringTok{ }\NormalTok{Trans.oceanic }\OperatorTok{+}
\StringTok{      }\NormalTok{Fisheries }\OperatorTok{+}\StringTok{ }\NormalTok{commercial }\OperatorTok{+}\StringTok{ }\NormalTok{subsistence }\OperatorTok{+}\StringTok{ }\NormalTok{Gamefish }\OperatorTok{+}\StringTok{ }\NormalTok{Aquarium }\OperatorTok{+}\StringTok{ }\NormalTok{Human.Uses}\OperatorTok{+}\StringTok{ }\NormalTok{D1 }\OperatorTok{+}\StringTok{ }\NormalTok{D2 }\OperatorTok{+}\StringTok{ }\NormalTok{D3 }\OperatorTok{+}
\StringTok{      }\NormalTok{log_size }\OperatorTok{+}\StringTok{ }\NormalTok{Rep_Strategy, }\DataTypeTok{data =}\NormalTok{ dd, }\DataTypeTok{family =} \StringTok{"binomial"}\NormalTok{)}

\NormalTok{firstorder.mod }
\end{Highlighting}
\end{Shaded}

\begin{verbatim}
## 
## Call:  glm(formula = Data.Deficient ~ Deepwater + Coastal + Pelagic + 
##     Benthic + BrackishFreshwater + Tropical + Temperate + Global + 
##     Pacific + Atlantic + Indian + Arctic + Antarctic + Trans.oceanic + 
##     Fisheries + commercial + subsistence + Gamefish + Aquarium + 
##     Human.Uses + D1 + D2 + D3 + log_size + Rep_Strategy, family = "binomial", 
##     data = dd)
## 
## Coefficients:
##               (Intercept)                  Deepwater  
##                   1.89798                    0.42050  
##                   Coastal                    Pelagic  
##                  -0.29459                   -0.09330  
##                   Benthic         BrackishFreshwater  
##                   0.65265                   -1.53095  
##                  Tropical                  Temperate  
##                   0.22480                   -0.02920  
##                    Global                    Pacific  
##                   0.38262                    0.17610  
##                  Atlantic                     Indian  
##                   0.10587                   -0.86275  
##                    Arctic                  Antarctic  
##                   1.78031                         NA  
##             Trans.oceanic                 Fisheries1  
##                  -0.33777                   -0.60877  
##                commercial                subsistence  
##                   0.12554                    0.04848  
##                  Gamefish                   Aquarium  
##                  -0.47694                    0.98909  
##               Human.Uses1                        D11  
##                   0.29409                   -0.48294  
##                       D21                        D31  
##                  -0.28874                   -0.81920  
##                  log_size  Rep_StrategyOvoviviparous  
##                  -0.51819                    0.02542  
##    Rep_StrategyViviparous      Rep_StrategyOviparous  
##                   0.36607                   -0.11306  
## 
## Degrees of Freedom: 500 Total (i.e. Null);  474 Residual
## Null Deviance:       663 
## Residual Deviance: 556.6     AIC: 610.6
\end{verbatim}

\begin{Shaded}
\begin{Highlighting}[]
\NormalTok{### note that we did try to include "Family" as a random variable; however, it was "rank deficient", which means that it was correlated with multiple other variables and therefore redundant. However, code is below}

\NormalTok{firstorder.mod.re <-}\StringTok{ }\KeywordTok{glmer}\NormalTok{(Data.Deficient }\OperatorTok{~}\StringTok{ }\NormalTok{Deepwater }\OperatorTok{+}\StringTok{ }\NormalTok{Coastal }\OperatorTok{+}\StringTok{ }\NormalTok{Pelagic }\OperatorTok{+}\StringTok{ }\NormalTok{Benthic }\OperatorTok{+}\StringTok{ }\NormalTok{BrackishFreshwater }\OperatorTok{+}\StringTok{ }
\StringTok{      }\NormalTok{Tropical }\OperatorTok{+}\StringTok{ }\NormalTok{Temperate }\OperatorTok{+}\StringTok{ }
\StringTok{      }\NormalTok{Global }\OperatorTok{+}\StringTok{ }\NormalTok{Pacific }\OperatorTok{+}\StringTok{ }\NormalTok{Atlantic }\OperatorTok{+}\StringTok{ }\NormalTok{Indian }\OperatorTok{+}\StringTok{ }\NormalTok{Arctic }\OperatorTok{+}\StringTok{ }\NormalTok{Antarctic }\OperatorTok{+}\StringTok{ }\NormalTok{Trans.oceanic }\OperatorTok{+}
\StringTok{      }\NormalTok{Fisheries }\OperatorTok{+}\StringTok{ }\NormalTok{commercial }\OperatorTok{+}\StringTok{ }\NormalTok{subsistence }\OperatorTok{+}\StringTok{ }\NormalTok{Gamefish }\OperatorTok{+}\StringTok{ }\NormalTok{Aquarium }\OperatorTok{+}\StringTok{ }\NormalTok{Vulnerability }\OperatorTok{+}\StringTok{ }
\StringTok{      }\NormalTok{log_size }\OperatorTok{+}\StringTok{ }\NormalTok{Rep_Strategy }\OperatorTok{+}
\StringTok{        }\NormalTok{(}\DecValTok{1}\OperatorTok{|}\NormalTok{Order), }\DataTypeTok{data =}\NormalTok{ dd, }\DataTypeTok{family =} \StringTok{"binomial"}\NormalTok{)}
\end{Highlighting}
\end{Shaded}

\begin{verbatim}
## fixed-effect model matrix is rank deficient so dropping 1 column / coefficient
\end{verbatim}

\begin{verbatim}
## singular fit
\end{verbatim}

\begin{Shaded}
\begin{Highlighting}[]
\NormalTok{### Potential second order model}
\NormalTok{secondorder.mod <-}\StringTok{ }\KeywordTok{glm}\NormalTok{(Data.Deficient }\OperatorTok{~}\StringTok{ }\NormalTok{Deepwater }\OperatorTok{+}\StringTok{ }\NormalTok{Coastal }\OperatorTok{+}\StringTok{ }\NormalTok{Pelagic }\OperatorTok{+}\StringTok{ }\NormalTok{Benthic }\OperatorTok{+}\StringTok{ }\NormalTok{BrackishFreshwater }\OperatorTok{+}
\StringTok{      }\NormalTok{Tropical }\OperatorTok{+}\StringTok{ }\NormalTok{Temperate }\OperatorTok{+}\StringTok{ }
\StringTok{      }\NormalTok{Global }\OperatorTok{+}\StringTok{ }\NormalTok{Pacific }\OperatorTok{+}\StringTok{ }\NormalTok{Atlantic }\OperatorTok{+}\StringTok{ }\NormalTok{Indian }\OperatorTok{+}\StringTok{ }\NormalTok{Arctic }\OperatorTok{+}\StringTok{ }\NormalTok{Antarctic }\OperatorTok{+}\StringTok{ }\NormalTok{Trans.oceanic }\OperatorTok{+}
\StringTok{      }\NormalTok{Fisheries }\OperatorTok{+}\StringTok{ }\NormalTok{commercial }\OperatorTok{+}\StringTok{ }\NormalTok{subsistence }\OperatorTok{+}\StringTok{ }\NormalTok{Gamefish }\OperatorTok{+}\StringTok{ }\NormalTok{Aquarium }\OperatorTok{+}\StringTok{ }\NormalTok{Human.Uses}\OperatorTok{+}\StringTok{ }\NormalTok{D1 }\OperatorTok{+}\StringTok{ }\NormalTok{D2 }\OperatorTok{+}\StringTok{ }\NormalTok{D3 }\OperatorTok{+}
\StringTok{      }\NormalTok{log_size }\OperatorTok{+}\StringTok{ }\NormalTok{Rep_Strategy }\OperatorTok{+}\StringTok{ }\NormalTok{Coastal}\OperatorTok{*}\NormalTok{Fisheries }\OperatorTok{+}\StringTok{ }\NormalTok{Pelagic}\OperatorTok{*}\NormalTok{Fisheries }\OperatorTok{+}\StringTok{ }\NormalTok{Benthic}\OperatorTok{*}\NormalTok{Fisheries }\OperatorTok{+}\StringTok{ }\NormalTok{BrackishFreshwater}\OperatorTok{*}\NormalTok{Fisheries, }\DataTypeTok{data =}\NormalTok{ dd, }\DataTypeTok{family =} \StringTok{"binomial"}\NormalTok{)}

\NormalTok{secondorder.mod }
\end{Highlighting}
\end{Shaded}

\begin{verbatim}
## 
## Call:  glm(formula = Data.Deficient ~ Deepwater + Coastal + Pelagic + 
##     Benthic + BrackishFreshwater + Tropical + Temperate + Global + 
##     Pacific + Atlantic + Indian + Arctic + Antarctic + Trans.oceanic + 
##     Fisheries + commercial + subsistence + Gamefish + Aquarium + 
##     Human.Uses + D1 + D2 + D3 + log_size + Rep_Strategy + Coastal * 
##     Fisheries + Pelagic * Fisheries + Benthic * Fisheries + BrackishFreshwater * 
##     Fisheries, family = "binomial", data = dd)
## 
## Coefficients:
##                   (Intercept)                      Deepwater  
##                      1.917529                       0.379405  
##                       Coastal                        Pelagic  
##                      0.023509                       0.132359  
##                       Benthic             BrackishFreshwater  
##                      0.665089                     -15.219208  
##                      Tropical                      Temperate  
##                      0.345438                       0.059848  
##                        Global                        Pacific  
##                      0.476997                       0.142708  
##                      Atlantic                         Indian  
##                      0.069941                      -0.970620  
##                        Arctic                      Antarctic  
##                      1.541260                             NA  
##                 Trans.oceanic                     Fisheries1  
##                     -0.651753                       0.066194  
##                    commercial                    subsistence  
##                      0.204895                      -0.073075  
##                      Gamefish                       Aquarium  
##                     -0.466649                       0.977947  
##                   Human.Uses1                            D11  
##                      0.286262                      -0.488341  
##                           D21                            D31  
##                     -0.323358                      -0.772454  
##                      log_size      Rep_StrategyOvoviviparous  
##                     -0.559206                       0.007368  
##        Rep_StrategyViviparous          Rep_StrategyOviparous  
##                      0.509051                      -0.117778  
##            Coastal:Fisheries1             Pelagic:Fisheries1  
##                     -1.169136                      -0.389077  
##            Benthic:Fisheries1  BrackishFreshwater:Fisheries1  
##                      0.086389                      14.057666  
## 
## Degrees of Freedom: 500 Total (i.e. Null);  470 Residual
## Null Deviance:       663 
## Residual Deviance: 548.4     AIC: 610.4
\end{verbatim}

\subsubsection{Now to visualize a coefficient plots to see what is
significant!}\label{now-to-visualize-a-coefficient-plots-to-see-what-is-significant}

\begin{Shaded}
\begin{Highlighting}[]
\CommentTok{# Pull out plot data}
\NormalTok{coefs <-}\StringTok{ }\KeywordTok{plot_model}\NormalTok{(firstorder.mod, }\DataTypeTok{transform =} \OtherTok{NULL}\NormalTok{)}

\CommentTok{# Making custom figure}
\NormalTok{coefs}\OperatorTok{$}\NormalTok{data }\OperatorTok
\StringTok{  }\KeywordTok{mutate}\NormalTok{(}\DataTypeTok{term =} \KeywordTok{factor}\NormalTok{(term, }\DataTypeTok{levels =}\NormalTok{ term)) }\OperatorTok
\StringTok{  }\CommentTok{# Color by sign of change, transparency by p-value}
\StringTok{  }\KeywordTok{ggplot}\NormalTok{(}\KeywordTok{aes}\NormalTok{(}\DataTypeTok{x =}\NormalTok{ estimate,}
             \DataTypeTok{y =} \KeywordTok{as.numeric}\NormalTok{(term),}
             \DataTypeTok{color =} \KeywordTok{as.factor}\NormalTok{(}\KeywordTok{sign}\NormalTok{(estimate)),}
             \DataTypeTok{alpha =} \KeywordTok{as.factor}\NormalTok{(p.value }\OperatorTok{<}\StringTok{ }\FloatTok{0.05}\NormalTok{))) }\OperatorTok{+}
\StringTok{  }\KeywordTok{geom_point}\NormalTok{() }\OperatorTok{+}\StringTok{ }
\StringTok{  }\CommentTok{# Add CIs around point}
\StringTok{  }\KeywordTok{geom_segment}\NormalTok{(}\KeywordTok{aes}\NormalTok{(}\DataTypeTok{x =}\NormalTok{ conf.low,}
                   \DataTypeTok{xend  =}\NormalTok{ conf.high,}
                   \DataTypeTok{yend =} \KeywordTok{as.numeric}\NormalTok{(term))) }\OperatorTok{+}
\StringTok{  }\KeywordTok{geom_vline}\NormalTok{(}\DataTypeTok{xintercept =} \DecValTok{0}\NormalTok{) }\OperatorTok{+}
\StringTok{  }\CommentTok{# Adding labelled y axis}
\StringTok{  }\KeywordTok{scale_y_continuous}\NormalTok{(}\DataTypeTok{labels =} \KeywordTok{as.character}\NormalTok{(coefs}\OperatorTok{$}\NormalTok{data}\OperatorTok{$}\NormalTok{term),}
                     \DataTypeTok{breaks =} \KeywordTok{seq}\NormalTok{(}\DecValTok{1}\NormalTok{, }\DecValTok{26}\NormalTok{, }\DecValTok{1}\NormalTok{)) }\OperatorTok{+}
\StringTok{  }\KeywordTok{scale_alpha_discrete}\NormalTok{(}\DataTypeTok{range =} \KeywordTok{c}\NormalTok{(.}\DecValTok{5}\NormalTok{, }\DecValTok{1}\NormalTok{)) }\OperatorTok{+}
\StringTok{  }\KeywordTok{xlab}\NormalTok{(}\StringTok{"Log Odds Ratio of Being DD (Higher Values = Greater Chance)"}\NormalTok{) }\OperatorTok{+}
\StringTok{  }\KeywordTok{ylab}\NormalTok{(}\StringTok{"Variables"}\NormalTok{)}
\end{Highlighting}
\end{Shaded}

\begin{verbatim}
## Warning: Using alpha for a discrete variable is not advised.
\end{verbatim}

\includegraphics{DataDeficiency_analysis_files/figure-latex/unnamed-chunk-3-1.pdf}

\begin{Shaded}
\begin{Highlighting}[]
\NormalTok{############# REPEAT FOR SECOND ORDER MODEL #############}

\CommentTok{# Pull out plot data}
\NormalTok{coefs2 <-}\StringTok{ }\KeywordTok{plot_model}\NormalTok{(secondorder.mod, }\DataTypeTok{transform =} \OtherTok{NULL}\NormalTok{)}

\CommentTok{# Making custom figure}
\NormalTok{coefs2}\OperatorTok{$}\NormalTok{data }\OperatorTok
\StringTok{  }\KeywordTok{mutate}\NormalTok{(}\DataTypeTok{term =} \KeywordTok{factor}\NormalTok{(term, }\DataTypeTok{levels =}\NormalTok{ term)) }\OperatorTok
\StringTok{  }\CommentTok{# Color by sign of change, transparency by p-value}
\StringTok{  }\KeywordTok{ggplot}\NormalTok{(}\KeywordTok{aes}\NormalTok{(}\DataTypeTok{x =}\NormalTok{ estimate,}
             \DataTypeTok{y =} \KeywordTok{as.numeric}\NormalTok{(term),}
             \DataTypeTok{color =} \KeywordTok{as.factor}\NormalTok{(}\KeywordTok{sign}\NormalTok{(estimate)),}
             \DataTypeTok{alpha =} \KeywordTok{as.factor}\NormalTok{(p.value }\OperatorTok{<}\StringTok{ }\FloatTok{0.05}\NormalTok{))) }\OperatorTok{+}
\StringTok{  }\KeywordTok{geom_point}\NormalTok{() }\OperatorTok{+}\StringTok{ }
\StringTok{  }\CommentTok{# Add CIs around point}
\StringTok{  }\KeywordTok{geom_segment}\NormalTok{(}\KeywordTok{aes}\NormalTok{(}\DataTypeTok{x =}\NormalTok{ conf.low,}
                   \DataTypeTok{xend  =}\NormalTok{ conf.high,}
                   \DataTypeTok{yend =} \KeywordTok{as.numeric}\NormalTok{(term))) }\OperatorTok{+}
\StringTok{  }\KeywordTok{geom_vline}\NormalTok{(}\DataTypeTok{xintercept =} \DecValTok{0}\NormalTok{) }\OperatorTok{+}
\StringTok{  }\CommentTok{# Adding labelled y axis}
\StringTok{  }\KeywordTok{scale_y_continuous}\NormalTok{(}\DataTypeTok{labels =} \KeywordTok{as.character}\NormalTok{(coefs2}\OperatorTok{$}\NormalTok{data}\OperatorTok{$}\NormalTok{term),}
                     \DataTypeTok{breaks =} \KeywordTok{seq}\NormalTok{(}\DecValTok{1}\NormalTok{, }\DecValTok{30}\NormalTok{, }\DecValTok{1}\NormalTok{)) }\OperatorTok{+}
\StringTok{  }\KeywordTok{scale_alpha_discrete}\NormalTok{(}\DataTypeTok{range =} \KeywordTok{c}\NormalTok{(.}\DecValTok{5}\NormalTok{, }\DecValTok{1}\NormalTok{)) }\OperatorTok{+}
\StringTok{  }\KeywordTok{xlab}\NormalTok{(}\StringTok{"Log Odds Ratio of Being DD (Higher Values = Greater Chance)"}\NormalTok{) }\OperatorTok{+}
\StringTok{  }\KeywordTok{ylab}\NormalTok{(}\StringTok{"Variables"}\NormalTok{)}
\end{Highlighting}
\end{Shaded}

\begin{verbatim}
## Warning: Using alpha for a discrete variable is not advised.
\end{verbatim}

\includegraphics{DataDeficiency_analysis_files/figure-latex/unnamed-chunk-3-2.pdf}

\section{Results}\label{results}

Here we see a few significant relationships:

\emph{Biology}\\
\textbf{Size} negative correlation with DD\\
Reproductive strategy

\emph{Human Use}\\
Human use in general (any use)\\
\textbf{Type of fishery} (\textbf{commercial - negative correlation},
gamefish, aquaculture)\\
Fisheries (any fishery)

\emph{Ecology}\\
\textbf{Geographic range} positive corr. with Arctic Ocean (note small
sample size); negative corr. with Indian Ocean\\
Latitude/temperature (tropical/temperate)\\
\textbf{Habitat} (\textbf{benthic - positive correlation}, deepwater,
pelagic, coastal, \textbf{freshwater - negative correlation})\\
\textbf{Depth range} (\textbf{0-200 m - negative correlation}, 201-1000
m, \textbf{1000+ m- negative correlation})

\emph{Let's summarize what this all means. In general, we are seeing
that larger sharks are less likely to be data deficient, as are species
targeted by fisheries specifically, rather than just human use in
general. In terms of habitat, sharks are less likely to be data
deficient if they are found in the Indian Ocean, in freshwater, and
either in the photic zone or at depths of greater than 1000 m. Note that
there may be an influence of sample size in some of these cases.}

\emph{In contrast, highly vulnerable species are more likely to be data
deficient. Sharks in the Arctic are also more likely to be data
deficient, although we have a small sample size for this. Interestingly,
although sharks at depths of greater than 1000 m are LESS likely to be
data deficient, benthic species are likely to be. These two
characteristics are not necessarily correlated, as benthic species could
reside at all depths, depending on whether they are coastal or deep
water.}

\subsubsection{Let's visualize our significant
results}\label{lets-visualize-our-significant-results}

\begin{Shaded}
\begin{Highlighting}[]
\CommentTok{# We will need to backtransform in order to get the relationships on the real scale, rather than the transformed scale}

\NormalTok{## extract all coefficients, make into a data frame}
\NormalTok{fullcoeffs <-}\StringTok{ }\KeywordTok{summary}\NormalTok{(firstorder.mod)}\OperatorTok{$}\NormalTok{coefficients }\OperatorTok\StringTok{ }\KeywordTok{as.data.frame}\NormalTok{(.) }\OperatorTok\StringTok{ }\KeywordTok{rownames_to_column}\NormalTok{()}
\NormalTok{coef_conf <-}\StringTok{ }\KeywordTok{confint}\NormalTok{(firstorder.mod) }\OperatorTok\StringTok{ }\KeywordTok{as.data.frame}\NormalTok{(.) }\OperatorTok\StringTok{ }\KeywordTok{rownames_to_column}\NormalTok{()}
\end{Highlighting}
\end{Shaded}

\begin{verbatim}
## Waiting for profiling to be done...
\end{verbatim}

\begin{Shaded}
\begin{Highlighting}[]
\CommentTok{# adds confidence intervals}
\NormalTok{coef_table <-}\StringTok{ }\KeywordTok{left_join}\NormalTok{(fullcoeffs, coef_conf) }\OperatorTok
\StringTok{  }\KeywordTok{rename}\NormalTok{(}\StringTok{"Coef"}\NormalTok{ =}\StringTok{ "rowname"}\NormalTok{)}
\end{Highlighting}
\end{Shaded}

\begin{verbatim}
## Joining, by = "rowname"
\end{verbatim}

\begin{Shaded}
\begin{Highlighting}[]
\NormalTok{coef_table}
\end{Highlighting}
\end{Shaded}

\begin{verbatim}
##                         Coef    Estimate Std. Error     z value
## 1                (Intercept)  1.89798307  0.9993378  1.89924074
## 2                  Deepwater  0.42049906  0.3380793  1.24378840
## 3                    Coastal -0.29459044  0.2454618 -1.20014767
## 4                    Pelagic -0.09329568  0.4578282 -0.20377878
## 5                    Benthic  0.65265086  0.2684606  2.43108664
## 6         BrackishFreshwater -1.53095099  0.6895971 -2.22006590
## 7                   Tropical  0.22479967  0.3370047  0.66705199
## 8                  Temperate -0.02920161  0.3490005 -0.08367213
## 9                     Global  0.38262148  0.4423963  0.86488396
## 10                   Pacific  0.17609809  0.3135966  0.56154344
## 11                  Atlantic  0.10586736  0.3687716  0.28708110
## 12                    Indian -0.86274559  0.3191445 -2.70330718
## 13                    Arctic  1.78031054  1.6017366  1.11148768
## 14             Trans.oceanic -0.33777067  0.5825540 -0.57981004
## 15                Fisheries1 -0.60877476  0.3589603 -1.69593906
## 16                commercial  0.12554128  0.4514905  0.27805963
## 17               subsistence  0.04848245  0.5059173  0.09583077
## 18                  Gamefish -0.47693959  0.4269441 -1.11710074
## 19                  Aquarium  0.98909135  0.5909572  1.67371075
## 20               Human.Uses1  0.29409053  0.3110277  0.94554448
## 21                       D11 -0.48293789  0.2346903 -2.05776638
## 22                       D21 -0.28874109  0.3129482 -0.92264810
## 23                       D31 -0.81920015  0.3160997 -2.59158811
## 24                  log_size -0.51819204  0.2172108 -2.38566475
## 25 Rep_StrategyOvoviviparous  0.02542230  0.3176824  0.08002427
## 26    Rep_StrategyViviparous  0.36607452  0.4902662  0.74668517
## 27     Rep_StrategyOviparous -0.11305613  0.3451894 -0.32751916
##       Pr(>|z|)       2.5 %      97.5 %
## 1  0.057532831 -0.04375083  3.88483616
## 2  0.213577452 -0.23943305  1.08983071
## 3  0.230081994 -0.77736602  0.18647321
## 4  0.838526378 -1.03029856  0.78043045
## 5  0.015053614  0.12954257  1.18388478
## 6  0.026414294 -3.07980864 -0.29835210
## 7  0.504738928 -0.43520667  0.89001141
## 8  0.933317117 -0.71528740  0.65744467
## 9  0.387102483 -0.48334098  1.25528131
## 10 0.574427122 -0.44244928  0.79055118
## 11 0.774050224 -0.62224038  0.82751057
## 12 0.006865327 -1.50144286 -0.24659331
## 13 0.266358489 -1.70307878  5.27630354
## 14 0.562042727 -1.49944902  0.79389022
## 15 0.089897421 -1.31718319  0.09313587
## 16 0.780966581 -0.78177290  1.00015193
## 17 0.923654977 -0.97496978  1.02334685
## 18 0.263951253 -1.34830851  0.33790352
## 19 0.094187459 -0.18228542  2.16021474
## 20 0.344380975 -0.31490342  0.90711437
## 21 0.039612564 -0.94339201 -0.02184768
## 22 0.356190618 -0.91224539  0.31851914
## 23 0.009553407 -1.45013595 -0.20780617
## 24 0.017048284 -0.95129682 -0.09759334
## 25 0.936217954 -0.59736545  0.65065860
## 26 0.455253622 -0.60413074  1.32353381
## 27 0.743275255 -0.79193395  0.56406270
\end{verbatim}

\begin{Shaded}
\begin{Highlighting}[]
\NormalTok{########### Evan's Help ####}
\KeywordTok{library}\NormalTok{(emmeans) }\CommentTok{#computing estimated marginal means (least square means)}
\CommentTok{#Least square means are means for groups that are adjusted for means of other factors in the model (allows us to account for different sample sizes between different families, etc.)}

\NormalTok{logsize_means <-}\StringTok{ }\KeywordTok{emmeans}\NormalTok{(firstorder.mod, }\OperatorTok{~}\StringTok{ }\NormalTok{log_size, }
         \DataTypeTok{var =} \StringTok{"log_size"}\NormalTok{, }
         \DataTypeTok{at =} \KeywordTok{list}\NormalTok{(}\DataTypeTok{log_size =} \KeywordTok{seq}\NormalTok{(}\DecValTok{2}\NormalTok{, }\DecValTok{8}\NormalTok{, }\DataTypeTok{by =}\NormalTok{ .}\DecValTok{25}\NormalTok{)),}
         \DataTypeTok{type =} \StringTok{"response"}\NormalTok{)}

\CommentTok{# Plotting the figure}
\KeywordTok{data.frame}\NormalTok{(logsize_means) }\OperatorTok
\StringTok{  }\KeywordTok{ggplot}\NormalTok{(}\KeywordTok{aes}\NormalTok{(}\DataTypeTok{x =} \KeywordTok{exp}\NormalTok{(log_size), }\CommentTok{# Exponentiating log size}
             \DataTypeTok{y =}\NormalTok{ prob)) }\OperatorTok{+}
\StringTok{  }\KeywordTok{geom_ribbon}\NormalTok{(}\KeywordTok{aes}\NormalTok{(}\DataTypeTok{ymin =}\NormalTok{ asymp.LCL,}
              \DataTypeTok{ymax =}\NormalTok{ asymp.UCL),}
              \DataTypeTok{alpha =}\NormalTok{ .}\DecValTok{2}\NormalTok{) }\OperatorTok{+}
\StringTok{  }\KeywordTok{geom_line}\NormalTok{() }\OperatorTok{+}
\StringTok{  }\KeywordTok{theme_bw}\NormalTok{() }\OperatorTok{+}
\StringTok{  }\KeywordTok{xlab}\NormalTok{(}\StringTok{"Average species size (cm)"}\NormalTok{) }\OperatorTok{+}
\StringTok{  }\KeywordTok{ylab}\NormalTok{(}\StringTok{"Probability of Data Deficiency"}\NormalTok{)}
\end{Highlighting}
\end{Shaded}

\includegraphics{DataDeficiency_analysis_files/figure-latex/unnamed-chunk-4-1.pdf}

\begin{Shaded}
\begin{Highlighting}[]
\CommentTok{#  }
\CommentTok{# vulnerability_means <- emmeans(firstorder.mod, ~ Vulnerability, }
\CommentTok{#          var = "Vulnerability", }
\CommentTok{#          at = list(Vulnerability = seq(0, 4)),}
\CommentTok{#          type = "response")}
\CommentTok{# }
\CommentTok{# data.frame(vulnerability_means) %>%}
\CommentTok{#   ggplot(aes(x = Vulnerability,}
\CommentTok{#              y = prob,}
\CommentTok{#              group = Vulnerability,}
\CommentTok{#              ymin = asymp.LCL,}
\CommentTok{#              ymax = asymp.UCL)) +}
\CommentTok{#   geom_point() +}
\CommentTok{#   geom_errorbar(width = .1)+}
\CommentTok{#   theme_bw()}
\end{Highlighting}
\end{Shaded}

\begin{Shaded}
\begin{Highlighting}[]
\NormalTok{### Plot backtransformed values against all values of size in this data set (dd$Size vs. )}

\NormalTok{#### Habitat}
\CommentTok{#hab_coeffs <- plogis(fullcoeffs[c("Benthic", "Deepwater", "Pelagic", "Coastal", "BrackishFreshwater"),])}
\CommentTok{#barplot(hab_coeffs[,"Estimate"], ylab="Relative Probability of Data Deficiency")}

\CommentTok{# OR! }

\CommentTok{#hab_coeffs.2 <- fullcoeffs[c("Benthic", "Deepwater", "Pelagic", "Coastal", "BrackishFreshwater"),]}
\CommentTok{#barplot(hab_coeffs.2[,"Estimate"], ylab="Log Odds Ratio of Data Deficiency")}


\NormalTok{### Geography}
\CommentTok{#geo_coeffs <- plogis(fullcoeffs[c("Atlantic", "Arctic", "Global", "Indian", "Pacific", "Global"),]) #note that this is a matrix}
\CommentTok{#barplot(geo_coeffs[,"Estimate"], ylab="Relative Probability of Data Deficiency")}

\CommentTok{#geo_coeffs.2 <- fullcoeffs[c("Atlantic", "Arctic", "Global", "Indian", "Pacific", "Global"),] #note that this is a matrix}
\CommentTok{#barplot(geo_coeffs.2[,"Estimate"], ylab="Log Odds Ratio of Data Deficiency")}


\NormalTok{### Depth range}
\CommentTok{#depth_coeffs <- plogis(fullcoeffs[c("D11", "D21", "D31"),])}
\CommentTok{#depth_coeffs}
\CommentTok{#barplot(depth_coeffs[,"Estimate"], names.arg=c("0-200m", "201-1000m", "1000+"), horiz=TRUE, xlab="Relative Probability of Data Deficiency", ylab="Depth")}

\CommentTok{#depth_coeffs.2 <- fullcoeffs[c("D11", "D21", "D31"),]}
\CommentTok{#barplot(depth_coeffs.2[,"Estimate"], names.arg=c("0-200m", "201-1000m", "1000+"), horiz=TRUE, xlab="Log Odds Ratio of Data Deficiency", ylab="Depth")}
\end{Highlighting}
\end{Shaded}


\end{document}
